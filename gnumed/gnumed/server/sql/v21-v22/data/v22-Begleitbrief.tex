%------------------------------------------------------------------
% Based on:
%
% Deutsche LaTeX Briefvorlage von Jan-Philip Gehrcke
% http://gehrcke.de -- jgehrcke@googlemail.com
% November 2009, Aktualisierung Januar 2013
%
% Stark angepaßt von Karsten.Hilbert@gmx.net für GNUmed.
%
% use with pdflatex, NOT xelatex
%
% License: GPL v2 or later
%------------------------------------------------------------------

\documentclass[
	paper=a4,
	BCOR=0mm,							% keine BindeCORrektur
	DIN,
	version=last,
	enlargefirstpage=on,
	fontsize=11pt,
	foldmarks=true,
	foldmarks=HP,
	parskip=full,						% Absätze mit Leerzeilen trennen, kein Einzug
	fromurl=false,						% explizit in der Fußzeile
	fromemail=false,					% explizit in der Fußzeile
	fromfax=false,						% explizit in der Fußzeile
	fromphone=false,					% explizit in der Fußzeile
	fromlogo=true,
	fromalign=center,
	fromrule=afteraddress,
	symbolicnames=true,
	headsepline=true,
	footsepline=true,
	pagenumber=footright
]{scrlttr2}

\usepackage{scrlayer-scrpage}			% Kopf- und Fußzeilen
\usepackage[cjkjis,graphics]{ucs}		% lots of UTF8 symbols, breaks with xelatex
\usepackage[utf8x]{inputenc}			% content is UTF8, breaks with xelatex
\usepackage[T1]{fontenc}				% fonts are T1
\usepackage[ngerman]{babel}				% Deutsch und Trennung
\usepackage{marvosym}					% Symbole: Handy, Telefon, E-Mail
\usepackage{textcomp}					% Symbole für Textmodus zum Escapen
\usepackage{lmodern}					% sans serif Latin Modern
\usepackage{graphicx}					% Grafiken laden (Logo und Unterschrift)
\usepackage[space]{grffile}				% besserer Zugriff auf Grafikdateien
\usepackage{lastpage}					% easy access to page number of last page
\usepackage{calc}						% \widthof (für signature)
\usepackage{microtype}					% improved text layout
\usepackage{hyperref}					% aktive URLs


% debugging
% - show all files used
\listfiles
% - show unorthodox/faulty/obsolete uses
\RequirePackage[l2tabu, orthodox]{nag}
% - mark overfull horizontal boxes
%\overfullrule=2cm


% Definiere Grundschrift: sans serif Latin Modern
\renewcommand*\familydefault{\sfdefault}


\begin{document}

% Kopfzeile ab zweiter Seite
\lehead{$<<if_not_empty::$<title::::>$//%s \\//::>>$$<lastname::::>$, $<firstname::::>$ ($<date_of_birth::%d.%B %Y::>$)}
\lohead{$<<if_not_empty::$<title::::>$//%s \\//::>>$$<lastname::::>$, $<firstname::::>$ ($<date_of_birth::%d.%B %Y::>$)}
\cehead{}
\cohead{}
\rehead{Seite \thepage/\pageref{LastPage}}
\rohead{Seite \thepage/\pageref{LastPage}}


% Absenderdetails
\setkomavar{fromname}{$<current_provider_name::%(title)s %(firstnames)s %(lastnames)s::>$, FA für $<current_provider_external_id::Fachgebiet//Ärztekammer::50>$}
\setkomavar{fromaddress}{\small
	$<<praxis::%(praxis)s, %(branch)s::120>>$\\
	$<<praxis_address::%(street)s %(number)s$<if_not_empty::%(subunit)s// (%(subunit)s)//::>$, %(postcode)s %(urb)s::60>>$
}
\setkomavar{fromlogo}{$<data_snippet::praxis-logo//\includegraphics[width=30mm]{%s}//image/png//.png::250>$}%$ -- this dollarsign de-confuses mcedit syntax coloring
\setkomavar{backaddress}{$<current_provider_firstnames::::1>$.$<current_provider_lastnames::::>$\\$<praxis_address::%(street)s %(number)s\\%(postcode)s %(urb)s::60>$}


% Geschäftszeile
\setkomavar{yourref}{$<free_text::Angabe in "Ihr Zeichen"::40>$}
\setkomavar{yourmail}{$<free_text::Angabe in "Ihr Schreiben vom"::20>$}
\setkomavar{myref}{$<free_text::Angabe in "Unser Zeichen"::40>$}
\setkomavar{date}{$<today::%d.%B %Y::50>$}
\setkomavar{place}{$<praxis_address::%(urb)s::120>$}


% Betreff, nämlich Patientendaten
\setkomavar{subject}[]{%
	$<<<ph_cfg::argumentsdivider//##//%% <%(name)s> set to [%(value)s]::>>>$
	$<<<if_not_empty::$<<free_text::Betreff für den Brief (Patientendaten werden automatisch angefügt.)::120>>$##%s\\##% no additional Betreff by user::>>>$
	$<<<ph_cfg::argumentsdivider//DEFAULT//%% <%(name)s> set to [%(value)s]::>>>$
	Patient: $<title::::>$ $<firstname::::>$ $<lastname::::>$ (geb $<date_of_birth::%d.%B %Y::>$)\\
	Adresse: $<adr_street::home::>$ $<adr_number::home::>$, $<adr_postcode::home::>$ $<adr_location::home::>$
}


% Unterschrift
\setkomavar{signature}{%
	$<data_snippet::autograph-$<<current_provider_lastnames::::>>$_$<<current_provider_firstnames::::>>$//\includegraphics[width=30mm]{%s}\\//image/png//.png::250>$\rule{\widthof{\tiny{}(Der Unterzeichner haftet nicht für unsignierte Änderungen des Inhalts.)}}{.1pt}\\
	$<<ph_cfg::argumentsdivider//##//%% <%(name)s> set to [%(value)s]::>>$
	$<<if_not_empty::$<current_provider_name::%(title)s::>$##$<current_provider_name::%(title)s %(firstnames)s %(lastnames)s::>$\\##$<current_provider_name::%(firstnames)s %(lastnames)s::>$\\::>>$
	$<<if_not_empty::$<current_provider_external_id::Fachgebiet//Ärztekammer::50>$##FA für %s\\##::>>$
	$<<ph_cfg::argumentsdivider//DEFAULT//%% <%(name)s> set to [%(value)s]::>>$
	{\tiny{}(Der Unterzeichner haftet nicht für unsignierte Änderungen des Inhalts.)}
}
\renewcommand*{\raggedsignature}{\raggedright}
\makeatletter
\@setplength{sigbeforevskip}{1.8cm}							% Definiere vertikalen Abstand vor der Unterschrift
\makeatother


% Fußzeile 1.Seite
\setkomavar{firstfoot}{%
	\rule{\textwidth}{.3pt}
	\parbox[t]{\textwidth}{%
		\tiny
		\begin{tabular}[t]{ll}%
			\Telefon{}       & $<praxis_comm::workphone::60>$\\
			\FAX{}           & $<praxis_comm::fax::60>$\\
			\Email{}         & \href{mailto:$<praxis_comm::email::60>$}{$<praxis_comm::email::60>$}\\
			$<<ph_cfg::argumentsdivider//##//%% <%(name)s> set to [%(value)s]::>>$
			\ComputerMouse{} & $<<if_not_empty::$<praxis_comm::web::>$##\href{http://%s}{$<praxis_comm::web::60>$}##% no praxis web address available::>>$
			$<<ph_cfg::argumentsdivider//DEFAULT//%% <%(name)s> set to [%(value)s]::>>$
		\end{tabular}%
		\hfill
		\begin{tabular}[t]{ll}%
			$<<ph_cfg::argumentsdivider//##//%% <%(name)s> set to [%(value)s]::>>$
			\multicolumn{2}{l}{$<<if_not_empty::$<current_provider_external_id::Fachgebiet//Ärztekammer::50>$##FA für %s##::>>$}\\
			$<<ph_cfg::argumentsdivider//DEFAULT//%% <%(name)s> set to [%(value)s]::>>$
			BSNR       & $<praxis_id::KV-BSNR//KV//%(value)s::25>$\\
			LANR       & $<current_provider_external_id::KV-LANR//KV::25>$\\
			\multicolumn{2}{l}{GNUmed $<client_version::::>$ (\href{http://www.gnumed.org}{www.gnumed.org})}
		\end{tabular}%
		\hfill
		\begin{tabular}[t]{ll}%
			\multicolumn{2}{l}{$<praxis_id::Bankname//Bank//%(value)s::60>$}\\
			IBAN       & $<praxis_id::IBAN//Bank//%(value)s::30>$\\
			BIC        & $<praxis_id::BIC//Bank//%(value)s::30>$\\
			\multicolumn{2}{l}{Vorlage $<form_name_long::::60>$ v$<form_version::::20>$}
		\end{tabular}%
	}
}


% Fußzeile ab zweiter Seite
\lefoot{{\tiny{}$<current_provider::::>$, $<praxis::%(praxis)s: %(branch)s::120>$}}
\lofoot{{\tiny{}$<current_provider::::>$, $<praxis::%(praxis)s: %(branch)s::120>$}}
\cefoot{}
\cofoot{}
\refoot{{\tiny{}$<today::%d.%B %Y::50>$}}
\rofoot{{\tiny{}$<today::%d.%B %Y::50>$}}


% Definiere Brief und Empfaenger
\begin{letter}{%
	$<receiver_name::::>$\\
	$<receiver_street::::>$\ $<receiver_number::::>$\ $<receiver_subunit::::>$\\
	$<receiver_postcode::::>$\ $<receiver_location::::>$\\
	$<receiver_country::::>$
}


% Anrede
\opening{%
	$<<free_text::Anrede, z.B. "Sehr geehrte Kollegin" (ohne Komma am Ende)//$<receiver_name::::>$::140>>$,
}


% Brieftext
\selectlanguage{ngerman}
$<<free_text::Der eigentliche Brieftext (in LaTeX)::>>$


% Grussformel
\closing{%
	$<<free_text::Grußformel, z.B. "Mit freundlichen Grüßen" (ohne Komma am Ende)::140>>$,
}


% Anlagen
%\setkomavar*{enclseparator}[Anlage(n)]					% Titel für Anlagebereich
\encl{$<<free_text::Liste von Anlagen::300>>$}


% kein Verteiler
%\cc{}

\end{letter}

\end{document}
