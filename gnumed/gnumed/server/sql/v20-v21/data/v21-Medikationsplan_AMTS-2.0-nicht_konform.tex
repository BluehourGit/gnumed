%------------------------------------------------------------------
% Medikationsplan gemäß AMTS 2.0 für GNUmed (http://www.gnumed.de)
%
% License: GPL v2 or later
%
% Author: karsten.hilbert@gmx.net
% Thanks to: G.Hellmann
%
% requires pdflatex to be run with -recorder option
%------------------------------------------------------------------

% debugging
\listfiles
\errorcontextlines 10000

\documentclass[
	version=last,
	paper=landscape,
	paper=a4,
	DIV=9,									% help typearea find a good Satzspiegel
	BCOR=0mm,								% keine BindeCORrektur
	fontsize=12pt,							% per spec
	parskip=full,							% Absätze mit Leerzeilen trennen, kein Einzug
	headsepline=off,
	footsepline=off,
	titlepage=false
]{scrartcl}

%------------------------------------------------------------------
% packages
\usepackage{scrlayer-scrpage}				% Kopf- und Fußzeilen
\usepackage{geometry}						% setup margins
\usepackage{microtype}						% micro-adjustments to typesetting
\usepackage[cjkjis,graphics]{ucs}			% lots of UTF8 symbols, breaks with xelatex
\usepackage[T1]{fontenc}					% fonts are T1
\usepackage[ngerman]{babel}					% Deutsch und Trennung
\usepackage[utf8x]{inputenc}				% content is UTF8, breaks with xelatex
\usepackage{textcomp}						% Symbole für Textmodus zum Escapen
\usepackage{ragged2e}						% improved alignment, needed for raggedleft in header table cell
\usepackage{tabularx}						% bessere Tabellen
\usepackage{tabu}							% bessere Tabellen
\usepackage{longtable}						% Tabellen über mehrere Seiten
\usepackage{helvet}							% Arial alike Helvetica
\usepackage{lastpage}						% easy access to page number of last page
\usepackage{embedfile}						% store copy of data file for producing data matrix inside PDF
\usepackage{array}							% improved column styles
\usepackage[abspath]{currfile}				% generically refer to input file
\usepackage{graphicx}						% Grafiken laden (datamatrix)
\usepackage[space]{grffile}					% besserer Zugriff auf Grafikdateien
\usepackage[export]{adjustbox}				% improved options for \includegraphics
%\usepackage[ocgcolorlinks=true]{hyperref}	% aktive URLs, needs to be loaded last, most of the time
\usepackage{hyperref}						% aktive URLs, needs to be loaded last, most of the time

% debugging:
%\usepackage{showkeys}

%\usepackage{marvosym}					% Symbole: Handy, Telefon, E-Mail
%\usepackage{calc}						% \widthof (für signature)

%------------------------------------------------------------------
% setup:
% - debugging
%\tracingtabu=2
%\hypersetup{debug=true}


% - placeholder handler options
% switch on ellipsis handling:
$<ph_cfg::ellipsis//…//%% <%(name)s> set to [%(value)s]::>$


% - PDF metadata
\hypersetup{
	pdftitle = {Medikationsplan: $<name::%(firstnames)s %(lastnames)s::>$, $<date_of_birth::%d.%b %Y::>$},
	pdfauthor = {$<current_provider::::>$, $<praxis::%(branch)s, %(praxis)s::>$},
	pdfsubject = {Medikationsplan (AMTS, NICHT KONFORM)},
	pdfproducer = {GNUmed $<client_version::::>$, Vorlage $<form_name_long::::>$ ($<form_version::::>$)},
	pdfdisplaydoctitle = true
}


% - precise positioning of things to satisfy spec
\setlength{\tabcolsep}{0pt}
\setlength{\parskip}{0pt}
\setlength{\topskip}{0pt}
\setlength{\marginparwidth}{0pt}
\setlength{\marginparsep}{0pt}


% - page
\geometry{
	verbose,
	% debugging:
%	showframe,											% show page frames
%	showcrop,											% show crop marks
	a4paper,
	landscape,
	includehead=true,
	headheight=4cm,
	headsep=0pt,
	includefoot=true,									% make margins absolute
	footskip=1cm,										% per spec
	left=0.8cm,right=0.8cm,top=0.8cm,bottom=0.8cm		% per spec
}

% - font: Arial (Helvetica)
\renewcommand{\rmdefault}{phv}
\renewcommand{\sfdefault}{phv}


% - header = top part
\lohead{
	\upshape
	% debugging (with vertical lines):
	%\begin{tabu} to \textwidth {>{\raggedleft\arraybackslash}p{7cm}|X[-1,L]|@{\hspace{1cm}}|X[-1,R]|p{4.6cm}}
	\begin{tabu} to \textwidth {>{\raggedleft\arraybackslash}p{7cm}@{\hspace{1.5mm}}X[-1,L]@{\hspace{1cm}}X[-1,R]p{4.3cm}}
			{\fontsize{20pt}{22pt}\selectfont \textbf{\href{http://www.akdae.de/AMTS/Medikationsplan/}{Medikationsplan}}} \newline
			\ \newline
			{\fontsize{14pt}{16pt}\selectfont Seite \thepage\ von \pageref{LastPage}} \newline
			% as long as we aren't certified we can put here any image we want :-)
			\includegraphics[scale=5,max height=2.4cm,center]{$<patient_photo::%s//image/png//.png::>$}
		&
			% line 1
			\fontsize{14pt}{16pt}\selectfont
			$<ph_cfg::ellipsis//NONE//%% <%(name)s> set to [%(value)s]::>$		% switch off ellipsis handling so first range of name does not get one if too long
			für:\ \textbf{$<name::%(firstnames)s %(lastnames)s::1-37>$} \newline
			$<ph_cfg::ellipsis//…//%% <%(name)s> set to [%(value)s]::>$			% but reenable in case second line of name still too long so that we need an ellipsis
			% line 2
			$<name::\textbf{%(firstnames)s %(lastnames)s}::38-90>$\  \newline
			% line 3
			ausgedruckt von: $<current_provider::::30>$ \newline
			% line 4
			\href{http://$<praxis_comm::web::>$}{$<praxis::%(branch)s, %(praxis)s::50>$} \newline
			% line 5: use \mbox{TEXT} to prevent TEXT from getting hyphenated
			$<ph_cfg::argumentsdivider//||//%% <%(name)s> set to [%(value)s]::>$
			$<praxis_address::\href{http://www.openstreetmap.org}{\mbox{$<<praxis_address::%(street)s %(number)s %(subunit)s,%(postcode)s %(urb)s::55>>$}}::>$
			$<ph_cfg::argumentsdivider//DEFAULT//%% <%(name)s> set to [%(value)s]::>$
			\ \newline
			% line 6
			$<<praxis_comm::workphone//\href{tel:$<praxis_comm::workphone::>$}{$<praxis_comm::workphone::50>$}::>>$\  \newline
			% line 7
			$<<praxis_comm::email//{\fontsize{12pt}{14pt}\selectfont \href{mailto:$<praxis_comm::email::>$}{$<praxis_comm::email::80>$}}::>>$\ 
		&
			\fontsize{14pt}{16pt}\selectfont			% alle Zeilen
			% line 1
			geb.\ am: \textbf{$<date_of_birth::%d.%b %Y::>$}\newline
			% line 2
			\ \newline
			% line 3
			\ \newline
			% line 4			%$<allergy_state::::15>$
			Allergie: Seite \pageref{LastPage}\ \hyperref[AnchorAllergieDetails]{unten}\newline
			% line 5
			\ \newline											%Gewicht: 125 kg\newline		::25
			% line 6
			\ \newline											%schwanger / stillend\newline	::25
			% line 7: soll eigentlich linksbündig ...
			ausgedruckt am: $<today::%d.%b %Y::>$
		&
			\includegraphics[valign=t,raise=6ex,max height=4cm,max width=4cm,center]{$<<amts_png_file_current_page::::>>$}
%		\tabularnewline
	\end{tabu}
}
%\lehead{test}
%\cehead{}
%\cohead{}
%\rehead{}
%\rohead{}


% footer setup = bottom part
\lofoot{{
	\begin{tabular}[t]{p{6cm}p{17cm}p{5cm}}
		\hline
		{	% left side: Versionsinformationen
			\raggedright
			\fontsize{10pt}{12pt}\selectfont
			\upshape
			%DE-DE-Version 2.0 vom 15.12.2014
			NICHT KONFORM (2.0)
		} & {
			% middle part: Herstellerfeld
			\parbox[t][1cm][t]{17cm}{
				\centering
				GNUmed $<client_version::::>$ --- \href{http://www.gnumed.org}{www.gnumed.org}
			}
		} & {
			% right side: Freifeld, must be empty
			% debugging:
			%---Freifeld---
		}
	\end{tabular}
}}
\lefoot{{
	\begin{tabular}[t]{p{6cm}p{17cm}p{5cm}}
		\hline
		{	% left side: Versionsinformationen
			\raggedright
			\fontsize{10pt}{12pt}\selectfont
			\upshape
			%DE-DE-Version 2.0 vom 15.12.2014
			NICHT KONFORM (2.0)
		} & {
			% middle part: Herstellerfeld
			\parbox[t][1cm][t]{17cm}{
				\centering
				GNUmed $<client_version::::>$ --- \href{http://www.gnumed.org}{www.gnumed.org}
			}
		} & {
			% right side: Freifeld, must be empty
			% debugging:
			%---Freifeld---
		}
	\end{tabular}
}}
\cefoot{}
\cofoot{}
\refoot{}
\rofoot{}

%------------------------------------------------------------------
\begin{document}

% middle part: Medikationsliste
\begin{longtable} {|
	% column definition
	p{4cm}|									% Wirkstoff
	p{4.4cm}|								% Handelsname
	>{\RaggedLeft}p{1.8cm}|					% Stärke
	p{1.8cm}|								% Form
	p{0.8cm}|								% Dosierung: morgens
	p{0.8cm}|								% Dosierung: mittags
	p{0.8cm}|								% Dosierung: abends
	p{0.8cm}|								% Dosierung: zur Nacht
	p{2.0cm}|								% Einheit
	p{6.4cm}|								% Hinweise
	p{4.3cm}|								% Grund
}
	% Tabellenkopf:
	\hline
	\rule{0pt}{4.5mm}						% vertical strut to ensure 14pt doesn't touch top \hline
	\fontsize{14pt}{16pt}\selectfont \centering Wirkstoff &
	\fontsize{14pt}{16pt}\selectfont \centering Handelsname &
	\fontsize{14pt}{16pt}\selectfont \centering Stärke &
	\fontsize{14pt}{16pt}\selectfont \centering Form &
	\fontsize{14pt}{16pt}\selectfont \centering Mo &
	\fontsize{14pt}{16pt}\selectfont \centering Mi &
	\fontsize{14pt}{16pt}\selectfont \centering Ab &
	\fontsize{14pt}{16pt}\selectfont \centering zN &
	\fontsize{14pt}{16pt}\selectfont \centering Einheit &
	\fontsize{14pt}{16pt}\selectfont \centering Hinweise &
	\fontsize{14pt}{16pt}\selectfont \centering Grund
	\endhead
	% Tabellenende auf 1. und 2. Seite
	\endfoot
	% Tabellenende auf letzter (3.) Seite
	\endlastfoot
	% Tabelleninhalt:
\hline
$<current_meds_AMTS_enhanced::::>$
\end{longtable}

%------------------------------------------------------------------
% include data in PDF for easier processing:

% VCF of creator
\IfFileExists{$<praxis_vcf::::>$}{
	\embedfile[
		desc=01) digitale Visitenkarte des Erstellers des Medikationsplans,
		mimetype=text/vcf,
		ucfilespec=AMTS-Medikationsplan-Ersteller.vcf
	]{$<praxis_vcf::::>$}
}{\typeout{[$<praxis_vcf::::>$] not found}}

% LaTeX source code from which the PDF was produced
\embedfile[
	desc=02) LaTeX-Quellcode des Medikationsplans (übersetzbar mit "pdflatex -recorder -interaction=nonstopmode \currfilename\ "),
	mimetype=text/plain,
	ucfilespec=\currfilename
]{\currfileabspath}

% enhanced QR code
\IfFileExists{$<<amts_png_file_utf8::::>>$}{
	\embedfile[
		desc=03) QR-Code (Bild) für alle Seiten (lesbar mit "dmtxread -v $<<amts_png_file_utf8::::>>$ > $<<amts_png_file_utf8::::>>$.txt"),
		mimetype=image/png,
		ucfilespec=AMTS-QR-Code-utf8-alle-Seiten.png
	]{$<<amts_png_file_utf8::::>>$}
}{\typeout{[$<<amts_png_file_utf8::::>>$] (all pages) not found}}

\IfFileExists{$<<amts_data_file_utf8::::>>$}{
	\embedfile[
		desc=04) QR-Code (Daten) für alle Seiten (übersetzbar mit "dmtxwrite -e a -m 2 -f PNG -o $<<amts_data_file_utf8::::>>$.png -s s -v $<<amts_data_file_utf8::::>>$"),
		mimetype=image/png,
		ucfilespec=AMTS-QR-Code-utf8-alle-Seiten.txt
	]{$<<amts_data_file_utf8::::>>$}
}{\typeout{[$<<amts_data_file_utf8::::>>$] (all pages) not found}}

% per-page QR codes
% page 1
\IfFileExists{$<<amts_png_file_1::::>>$}{
	\embedfile[
		desc=05) QR-Code (Bild) für Seite 1 (lesbar mit "dmtxread -U -v $<<amts_png_file_1::::>>$ > $<<amts_png_file_1::::>>$.txt"),
		mimetype=image/png,
		ucfilespec=AMTS-QR-Code-Seite-1.png
	]{$<<amts_png_file_1::::>>$}
}{\typeout{[$<<amts_png_file_1::::>>$] (page 1) not found}}

\IfFileExists{$<<amts_data_file_1::::>>$}{
	\embedfile[
		desc=06) QR-Code (Daten) für Seite 1 (übersetzbar mit "dmtxwrite -e a -m 2 -f PNG -o $<<amts_data_file_1::::>>$.png -s s -v $<<amts_data_file_1::::>>$"),
		mimetype=text/plain,
		ucfilespec=AMTS-QR-Code-Daten-Seite-1.txt
	]{$<<amts_data_file_1::::>>$}
}{\typeout{[$<<amts_data_file_1::::>>$] (page 1) not found}}

% page 2
\IfFileExists{$<<amts_png_file_2::::>>$}{
	\embedfile[
		desc=07) QR-Code (Bild) für Seite 2 (lesbar mit "dmtxread -U -v $<<amts_png_file_1::::>>$ > $<<amts_png_file_2::::>>$.txt"),
		mimetype=image/png,
		ucfilespec=AMTS-QR-Code-Seite-2.png
	]{$<<amts_png_file_2::::>>$}
}{\typeout{[$<<amts_png_file_2::::>>$] (page 2) not found}}

\IfFileExists{$<<amts_data_file_2::::>>$}{
	\embedfile[
		desc=08) QR-Code (Daten) für Seite 2 (übersetzbar mit "dmtxwrite -e a -m 2 -f PNG -o $<<amts_data_file_2::::>>$.png -s s -v $<<amts_data_file_2::::>>$"),
		mimetype=text/plain,
		ucfilespec=AMTS-QR-Code-Daten-Seite-2.txt
	]{$<<amts_data_file_2::::>>$}
}{\typeout{[$<<amts_data_file_2::::>>$] (page 2) not found}}

% page 3
\IfFileExists{$<<amts_png_file_3::::>>$}{
	\embedfile[
		desc=09) QR-Code (Bild) für Seite 3 (lesbar mit "dmtxread -U -v $<<amts_png_file_1::::>>$ > $<<amts_png_file_3::::>>$.txt"),
		mimetype=image/png,
		ucfilespec=AMTS-QR-Code-Seite-3.png
	]{$<<amts_png_file_3::::>>$}
}{\typeout{[$<<amts_png_file_3::::>>$] (page 3) not found}}

\IfFileExists{$<<amts_data_file_3::::>>$}{
	\embedfile[
		desc=10) QR-Code (Daten) für Seite 3 (übersetzbar mit "dmtxwrite -e a -m 2 -f PNG -o $<<amts_data_file_3::::>>$.png -s s -v $<<amts_data_file_3::::>>$"),
		mimetype=text/plain,
		ucfilespec=AMTS-QR-Code-Daten-Seite-3.txt
	]{$<<amts_data_file_3::::>>$}
}{\typeout{[$<<amts_data_file_3::::>>$] (page 3) not found}}

% patient photo
\IfFileExists{$<patient_photo::%s//image/png//.png::>$}{
	\embedfile[
		desc=11) Patientenphoto,
		mimetype=image/png,
		ucfilespec=Patientenphoto.png
	]{$<patient_photo::%s//image/png//.png::>$}
}{\typeout{[$<patient_photo::%s//image/png//.png::>$] (patient photo) not found}}

%------------------------------------------------------------------

\end{document}
%------------------------------------------------------------------
