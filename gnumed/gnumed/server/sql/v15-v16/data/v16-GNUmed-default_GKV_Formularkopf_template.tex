% ----------------------------------------------------------------------
% Author: Christoph Becker, Karsten Hilbert
% License: GPL v2 or later
%
% **********************************************************************
% ************   WARNING   *********************************************
% **********************************************************************
%
% Note that the German legal situation DOES NOT ALLOW you to print
% prescriptions for "Kassenpatienten" with this GNUmed form template.
%
% **********************************************************************
% **********************************************************************

\documentclass[10pt,german]{letter}

\usepackage[utf8x]{inputenc}
\usepackage[T1]{fontenc}
\usepackage{ucs}
\usepackage{babel}
\usepackage[a4paper]{geometry}
\geometry{verbose}

% define page dimensions as needed
\pagestyle{empty}
\oddsidemargin 64mm		% (90mm vom linken Papierrand - 1 Inch LaTeX-Festrand links - 1mm Druckerrand)
\topmargin 27mm			% (52mm vom oberen Papierrand - 1 Inch LaTeX-Festrand oben)
\headheight 0mm
\headsep 0mm
\topskip 0mm
\footskip 0mm
%\textwidth146mm
%\textheight104mm
%\paperwidth148mm
%\paperheight105mm

% define where to put stuff
\newcount\ganzlinks
%\newcount\stempeloffsetx
%\newcount\stempeloffsety

\begin{document}
\setlength{\unitlength}{1mm}

% x-Koordinate tut nichts zur Sache (weil \oddsidemargin sie definiert)
% y-Koordinate schiebt d. 'Bild' nach unten
\begin{picture}(0,0)

\ganzlinks=-80			% 80mm links der rechten unteren Ecke des Stammdatenkastens

% for debugging:
%\put(0,0){.+ (0,0)} % Ursprungskoordinate (rechte untere Ecke KVK-Datenblock)
%\put(0,-47){.+ (0,-47)}
%\put(\ganzlinks,0){.+ (-80,0)}
%\put(\ganzlinks,-47){.+ (-80,-47)}

%----- KVK-Daten -----------
%\put(\ganzlinks,38){\texttt{*KVKKrankenkasse*}}
\put(\ganzlinks,30){\texttt{$<name::%(lastnames)s, %(firstnames)s::50>$}}
\put(-20,25){\texttt{$<date_of_birth::%d.%m.%Y::10>$}}
\put(\ganzlinks,25){\texttt{$<adr_street::home::25>$ $<adr_number::home::5>$}}
\put(\ganzlinks,20){\texttt{$<adr_postcode::home::5>$ $<adr_location::home::30>$}}
%\put(\ganzlinks,10){\texttt{*KVKKassenNr*}}
%\put(-56,10){\texttt{*KVKVersichertenNr*}}
%\put(-22,10){\texttt{*KVKStatus*}}
%\put(\ganzlinks,2){\texttt{*KVKZa-KZVNr*}}
%\put(-49,2){\texttt{*KVKVKgueltigbis*}}
\put(-24,2){\texttt{$<today::%d.%m.%Y::10>$}}
%----- KVK-Daten -----------


\put(\ganzlinks,-10){$<free_text::tex//Freier Formulareintrag::9999>$}


%----- Stempel ---------------
% da dieser in verschiedenen Forumularen an verschiedenen Stellen
% gesetzt wird, wird hier mit einem Offset gearbeitet:
% -> nur der Offset ist anzupassen
%\stempeloffsetx=17
%\stempeloffsety=-10

%\put(\stempeloffsetx,\stempeloffsety){\Large*KZVNr*}
%\advance\stempeloffsetx by 22 \advance\stempeloffsety by 3
%\put(\stempeloffsetx,\stempeloffsety){\scriptsize KZV}
%\advance\stempeloffsety by -3
%\put(\stempeloffsetx,\stempeloffsety){\small*Kobl-Tr*}
%\advance\stempeloffsetx by -22 \advance\stempeloffsety by -5
%\put(\stempeloffsetx,\stempeloffsety){\normalsize*Zahnarzt*}
%\advance\stempeloffsety by -5
%\put(\stempeloffsetx,\stempeloffsety){\normalsize*Zahnarztname*}
%\advance\stempeloffsety by -5
%\put(\stempeloffsetx,\stempeloffsety){\normalsize*ZahnarztStrasse*}
%\advance\stempeloffsety by -5
%\put(\stempeloffsetx,\stempeloffsety){\normalsize*ZahnarztPlZOrt*}
%----- Stempel ---------------

\end{picture}

\end{document}


% ----------------------------------------------------------------------
% This file is *heavily* based on work by:
%
% Christoph Becker (Dauner Str. 21, 53539 Kelberg, Germany, cgbecker@gmx.de)
%
% contributed to GNUmed on 4. Febr. 2005 
%
% Prinzip:
%
% Das gesamte Dokument besteht nur aus EINEM Bild (d.h. einer
% 'picture'-Umgebung).
%
% Alle Felder des Formulars werden als Bildelemente vom Typ Text
% platziert (Siehe Helmut Kopka, Latex - Eine Einfuehrung, Bd 1,
% Kap. 6, Bilder) und zwar relativ, bezogen auf die Ursprungs-
% koordinate des Bildes. Als Ursprungskoordinate wird die rechte
% untere Ecke des Rahmens mit den Versichertenkartendaten
% (Formularkopf) verwendet.
%
% Diese Ursprungskoordinate kann nicht an einem Papierrand liegen, weil
% LaTeX einen Seitenrand von 0 ignoriert. Daher ist es besser, bei
% Formularen einen leicht erkennbaren Ursprungspunkt im Formular zu
% benutzen. Die Bildgroesse kann (0,0) sein.
%
% Der KZV- bzw. KV-Stempel wird mit einem Offset eingestellt, hier:
%
% \newcount\stempeloffsetx \stempeloffsetx=20
% \newcount\stempeloffsety \stempeloffsety=-10
%
% Dadurch kann der 'Stempel' durch einfaches Kopieren auch in andere
% Kassenformulare uebernommen werden (AU, HKP usw.). Es muss dann
% lediglich der Offset geaendert werden.
%
% Das Formular wurde ursprünglich auf einem EPSON 630 LQ-Nadeldrucker
% unter Windows XP mit TeXnicCenter entwickelt. Wenn sich auf einem
% anderen Drucker oder unter Linux zeigt, dass die 
% Ursprungskoordinate nicht stimmt, so kann diese ueber
%
%	\oddsidemargin
%	\topmargin
%
% angepasst werden.
% ----------------------------------------------------------------------
