%% LyX 1.6.5 created this file.  For more info, see http://www.lyx.org/.
%% Do not edit unless you really know what you are doing.

%------------------------------------------------------------------
% Author: Karsten Hilbert
% License: GPL
%
% $Id: GNUmed-default_medication_list_template.tex,v 1.6 2010-01-21 08:47:11 ncq Exp $
% $Revision: 1.6 $
%------------------------------------------------------------------

\documentclass[english]{article}
\usepackage[T1]{fontenc}
\usepackage[utf8x]{inputenc}
\usepackage{hyperref}
%\usepackage{ucs}
%\usepackage{lmodern}
%\usepackage{times}
%\usepackage{amsmath}
\usepackage[a4paper]{geometry}
\geometry{verbose,lmargin=2.54cm}
\pagestyle{empty}
\setlength{\parskip}{\medskipamount}
\setlength{\parindent}{0pt}

\makeatletter

%%%%%%%%%%%%%%%%%%%%%%%%%%%%%% LyX specific LaTeX commands.
%% Because html converters don't know tabularnewline
\providecommand{\tabularnewline}{\\}

\makeatother

\usepackage{babel}

\begin{document}

\footnotetext {\href {http://www.gnumed.org}{GNUmed} $<client_version>$ -- \href {http://www.gnumed.org}{www.gnumed.org}}
\footnotetext {\tiny $<tex_escape::$Id: GNUmed-default_medication_list_template.tex,v 1.6 2010-01-21 08:47:11 ncq Exp $>$}

\noindent \begin{center}
{\LARGE Current Medication List}%
\par\end{center}

\noindent \begin{center}
{\large Printed $<today::%Y %B %d>$}
\par\end{center}{\large \par}

\noindent \begin{center}
{\large $<current_provider>$}
\par\end{center}{\large \par}

\noindent Patient: $<lastname>$, $<firstname>$

\noindent Birthdate: $<date_of_birth::%Y %B %d>$

{\footnotesize
	$<adr_street::home>$ $<adr_number::home>$ \\
	$<adr_postcode::home>$ $<adr_location::home>$
}


\vspace*{\medskipamount}


\noindent Medication list {\tiny (ordered by brand)}{\tiny \par}

\noindent \begin{tabular}{|l|l|l|l|}
\hline 
Drug & Strength & Regimen & Aim \tabularnewline
\hline
\hline 
$<current_meds::%(brand)s %(preparation)s & %(strength)s & %(schedule)s & {\footnotesize %(aim)s} \tabularnewline {\footnotesize (%(substance)s)} & \multicolumn{3}{l}{{\footnotesize %(notes)s}} \hfill \vline \tabularnewline \hline >$
\end{tabular}

% uncomment, if and as needed, the next lines to append text below the table
%{\footnotesize			% start smaller font
% <any line(s) of info you wish to be printed immediately below the table>
%}						% end smaller font


\vfill{}


\noindent Known allergies:

\begin{tabular}{|l|l|l|}
\hline 
Agent & Type & Reaction\tabularnewline
\hline
\hline 
$<allergies::%(descriptor)s & %(l10n_type)s & {\footnotesize %(reaction)s} \tabularnewline \hline >$
\end{tabular}

\medskip{}


\noindent \begin{flushright}
\texttt{\textsl{\footnotesize Practice Stamp / Physician Signature}}
\par\end{flushright}
\end{document}

%------------------------------------------------------------------
% $Log: GNUmed-default_medication_list_template.tex,v $
% Revision 1.6  2010-01-21 08:47:11  ncq
% - much improved layout
%
% Revision 1.5  2010/01/15 12:45:01  ncq
% - eventually include template file CVS version properly :-)
%
% Revision 1.4  2009/12/30 12:47:18  ncq
% - remove CVS tag inside TeX entirely
%
% Revision 1.3  2009/12/30 12:42:14  ncq
% - add missing \ before $
%
% Revision 1.2  2009/12/22 12:05:02  ncq
% - properly escape version strings
%
% Revision 1.1  2009/12/22 10:11:06  ncq
% - new
%
%